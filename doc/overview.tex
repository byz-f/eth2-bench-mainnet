\documentclass[twoside,twocolumn]{article}
\usepackage{graphicx}
\usepackage[sc]{mathpazo}
\usepackage[T1]{fontenc}
\linespread{1.05}
\usepackage{microtype}
\usepackage[english]{babel}
\usepackage[hmarginratio=1:1,top=32mm,columnsep=20pt]{geometry}
\usepackage[hang, small,labelfont=bf,up,textfont=it,up]{caption}
\usepackage{booktabs}
\usepackage{lettrine}
\usepackage{enumitem}
\setlist[itemize]{noitemsep}
\usepackage{abstract}
\renewcommand{\abstractnamefont}{\normalfont\bfseries}
\renewcommand{\abstracttextfont}{\normalfont\small\itshape}
\usepackage{titlesec}
\renewcommand\thesection{\Roman{section}}
\renewcommand\thesubsection{\roman{subsection}}
\titleformat{\section}[block]{\large\scshape\centering}{\thesection.}{1em}{}
\titleformat{\subsection}[block]{\large}{\thesubsection.}{1em}{}
\usepackage{fancyhdr}
\pagestyle{fancy}
\fancyhead{}
\fancyfoot{}
\fancyhead[C]{Ethereum 2.0 Client Metrics $\bullet$ \today}
\fancyfoot[RO,LE]{\thepage}
\usepackage{titling}
\usepackage{hyperref}

\setlength{\droptitle}{-4\baselineskip}
\pretitle{\begin{center}\Huge\bfseries}
\posttitle{\end{center}}
\title{Ethereum 2.0 Client Metrics}
\author{\textsc{Afri Schoedon, \href{https://github.com/q9f}{@q9f}}}\date{\today}

\begin{document}
\maketitle
\section{Introduction}
\lettrine[nindent=0em,lines=3]{E}thereum 2.0 will be a new blockchain protocol enabling -- amongst others -- horizontal scalability through sharding and transitioning the chain to a proof-of-stake consensus algorithm.\par

\begin{figure}[t]
	\centering
	\includegraphics[page=1,width=0.45\textwidth]{../res/plots.pdf}
	\caption{Lighthouse is depicted in orange. All data collected is displayed in this matrix: time running, slot height, blocks per second, database size, memory usage, and peer count.}
	\label{img:lh}
\end{figure}

\begin{figure}[t]
	\centering
	\includegraphics[page=2,width=0.45\textwidth]{../res/plots.pdf}
	\caption{Prysm is depicted in purple. All data collected is displayed in this matrix, same as figure \ref{img:lh}.}
	\label{img:pr}
\end{figure}

\begin{figure}[t]
	\centering
	\includegraphics[page=3,width=0.45\textwidth]{../res/plots.pdf}
	\caption{Teku is depicted in turquoise. All data collected is displayed in this matrix, same as figure \ref{img:lh}.}
	\label{img:tk}
\end{figure}

\begin{figure}[t]
	\centering
	\includegraphics[page=4,width=0.45\textwidth]{../res/plots.pdf}
	\caption{Nimbus is depicted in blue. All data collected is displayed in this matrix, same as figure \ref{img:lh}.}
	\label{img:nb}
\end{figure}

\begin{figure}[t]
	\centering
	\includegraphics[page=5,width=0.45\textwidth]{../res/plots.pdf}
	\caption{Synchronization progress over time.}
	\label{img:sync:prog}
\end{figure}

\begin{figure}[t]
	\centering
	\includegraphics[page=9,width=0.45\textwidth]{../res/plots.pdf}
	\caption{Synchronization speed over time.}
	\label{img:sync:sped}
\end{figure}

\begin{figure}[t]
	\centering
	\includegraphics[page=6,width=0.45\textwidth]{../res/plots.pdf}
	\caption{Database size over time.}
	\label{img:db}
\end{figure}

\begin{figure}[t]
	\centering
	\includegraphics[page=7,width=0.45\textwidth]{../res/plots.pdf}
	\caption{Resident memory usage over time.}
	\label{img:mem}
\end{figure}

\begin{figure}[t]
	\centering
	\includegraphics[page=8,width=0.45\textwidth]{../res/plots.pdf}
	\caption{Client's peer count over time.}
	\label{img:per}
\end{figure}

\subsection{Motivation}
None of the features that Ethereum 2.0 will bring are being implemented in established Ethereum 1.x clients such as Geth or Besu. Therefore, it's being worked on a new generation of core clients to power the beacon chain. None of these clients has ever been used in production before.\par

With the launch of the beacon chain supposedly happening in 2020, a second gathering of key metrics of four selected Ethereum 2.0 clients will be conducted, namely Lighthouse, Prysm, Teku, and Nimbus.\par

\subsection{Previous Benchmark}
In June 2020, a similar, preliminary benchmark was conducted\footnote{\href{https://github.com/q9f/eth2-bench-2020-06}{github.com/q9f/eth2-bench-2020-06}} gathering first insights into client metrics and getting feedback from the core-developer community.\par

Before diving into the results, please note the following.

\begin{itemize}
\item Most importantly, this work adds Nimbus to the list of profiled clients, allowing for comparing four client's metrics instead of three.
\item The numbers in this report are \textit{not} comparable with numbers in the previous report. This is mainly due to using different, dedicated, bare-metal hardware for gathering these numbers as compared to the virtual hosts used in the previous work.
\item Unfortunately, the previous report contained a methodological uncertainty. While all clients were run under same conditions doing a full synchronization, the Prysm client was apparently not built with optimized compiler settings\footnote{\href{https://github.com/prysmaticlabs/prysmaticlabs.gitbook.io/issues/9}{prysmaticlabs/prysmaticlabs.gitbook.io\#9}}. This has been revised and all clients are provided with release binaries.
\item Last but not least, this benchmark was conducted on the Altona testnet as compared to the Witti testnet in the last report. The Altona testnet has a different composition of validators and currently contains a little less blocks than the Witti testnet had in June 2020.
\end{itemize}

\subsection{Commented Data}
This article seeks to document the gathered metrics of different clients adhering to scientific methodology. It does not, however, intend to replace a peer-reviewed publication. It's simply a version of the data commented by the author.\par

The raw data is available on Github\footnote{\href{https://github.com/q9f/eth2-bench-2020-07}{github.com/q9f/eth2-bench-2020-07}} for further analysis.

\section{Clients}
\label{sec:cli}

Four clients are used for comparing key-performance metrics.\par

\textsc{Lighthouse} is an Ethereum 2.0 client developed by Sigma Prime\footnote{\href{https://github.com/sigp/lighthouse}{github.com/sigp/lighthouse}}. It's implemented in the Rust programming language. Data referring to the Lighthouse client is depicted in orange throughout this document (figure \ref{img:lh}).\par

\textsc{Prysm} is a beacon-chain implementation written in Go\footnote{\href{https://github.com/prysmaticlabs/prysm}{github.com/prysmaticlabs/prysm}}. It's being maintained by the Prysmatic Labs team. Data referring to the Prysm client is depicted in purple throughout this document (figure \ref{img:pr}).\par

\textsc{Teku} is an enterprise-grade Ethereum 2.0 client built by the PegaSys Engineering team\footnote{\href{https://github.com/PegaSysEng/teku}{github.com/PegaSysEng/teku}}. It's implemented in Java and data referring to the Teku client is depicted in turquoise throughout this document (figure \ref{img:tk}).\par

\textsc{Nimbus} is a beacon node implementation written in Nim built by the Status IM team\footnote{\href{https://github.com/status-im/nim-beacon-chain}{github.com/status-im/nim-beacon-chain}}. Data referring to the Nim Beacon Chain client is depicted in blue throughout this document (figure \ref{img:nb}).\par

Other clients implementing the Ethereum 2.0 protocol exist, namely ChainSafe Systems' \textsc{Lodestar}\footnote{\href{https://github.com/ChainSafe/lodestar}{github.com/ChainSafe/lodestar}}, Nethermind's \textsc{Cortex}\footnote{\href{https://github.com/NethermindEth/cortex}{github.com/NethermindEth/cortex}}, and the Ethereum Foundation's \textsc{Trinity}\footnote{\href{https://github.com/ethereum/trinity}{github.com/ethereum/trinity}}. Due to the different progress of implementing the protocol specification and core components, these clients were not considered for comparison, yet.\par

\section{Metadata}
The data is gathered on the Altona testnet\footnote{\href{https://github.com/goerli/altona}{github.com/goerli/altona}}. Altona is the third multi-client testnet launched with the four in section \ref{sec:cli} introduced clients as genesis validators.\par

At the time of collecting the metrics, the Altona testnet is based on \texttt{v0.12.1} of the Ethereum 2.0 beacon-chain specification. It contains approximately 120,000 slots and is run by 3,792 validators.\par

\subsection{Host Systems}
Four identical host systems have been installed for the sole purpose of the performance inspection. The host systems are dedicated bare-metal servers with an Ubuntu 20.04 LTS operating system kernel version \texttt{5.4.0-40-generic}.\par

The host machines are powered by an Intel Xeon E3-1240 v6 CPU with 8 cores. The available memory is 32 GB and the SSD disks allow for 250GB capacity.\par

\subsection{Client Versions}
All clients were compiled on July 16th, 2020, from the latest available source-code targeting the version \texttt{v0.12.1} of the Ethereum 2.0 specification.

\begin{itemize}
\item \textbf{Lighthouse}: version \texttt{lighthouse/0.1.2}, compiled from \texttt{master} branch at commit \texttt{fc5e6cbb} from July 16th, 2020, with Rust version \texttt{1.44.1}-stable through Cargo.
\item \textbf{Prysm}: compiled from \texttt{master} branch at commit \texttt{df738517} from July 16th, 2020, with Go version \texttt{1.13.8} through Bazel.
\item \textbf{Teku}: version \texttt{teku/v0.12.2-dev}, compiled from \texttt{master} branch at commit \texttt{04b0a00a} from July 16th, 2020, with Java version \texttt{14.0.1} through Gradle.
\item \textbf{Nimbus}: version \texttt{beacon node v0.5.0}, compiled from \texttt{devel} branch at commit \texttt{3dfbc311} from July 15th, 2020, with Nim version \texttt{1.2.2} through Make.
\end{itemize}

All clients were provided with a sufficient number of bootstrap nodes to ensure good connectivity and eliminate potential networking bottlenecks (compare section \ref{sec:perf} point \ref{sec:perf:p2p}).\par

\section{Performance}
\label{sec:perf}
This document does only inspect the performance metrics of beacon-chain node implementations. Other features such as running validator clients, bootstrap nodes, or other relevant tooling is disregarded for simplicity.

\subsection{Synchronization Metrics}
Figure \ref{img:sync:prog} displays the progress of synchronizing the four aforementioned clients. Notably, the Lighthouse client manages to fully synchronize all blocks and verify all signatures in a little less than 25 minutes with the Prysm client being \textit{on par} finishing the same task in just about the same time. Teku completes the same task in 1 hours and 26 minutes, whereas Nimbus requires 6 hours and 54 minutes to fully sync and verify the Altona beacon chain.\par

In addition, figure \ref{img:sync:sped} displays the same data but computing the synchronization speed in slots per second by taking the time required to fully catch up with the beacon-chain head. The plotted data points display a moving average over 60 seconds, the plotted line shows a moving average over 10 minutes. Lighthouse and Prysm lead the chart at an average of approximately 80 slots per second on the dedicated hardware.\par

The data at glance.
\begin{itemize}
\item \textbf{Lighthouse} synchronizes 122,105 slots in 1,495 seconds at an overall average speed of 81.676 slots per second.
\item \textbf{Prysm} catches up with 122,069 slots in 1,535 seconds at 79.524 slots per second.
\item \textbf{Teku} synchronizes 122,412 slots in 5,174 seconds at an average speed of 23.659 slots per second.
\item \textbf{Nimbus} catches up with 124,051 slots in 24,844 seconds at 4.9932 slots per second.
\end{itemize}

All clients do a full verification of all signatures during synchronization by default.\par

\subsection{Database Metrics}
Figure \ref{img:db} displays the database size in Bytes plotted over time of running the nodes. The patterns are left uncommented for the client developers to analyze.\par

The data at glance.
\begin{itemize}
\item \textbf{Teku} requires 66.3 MiB for 124,342 slots worth of beacon-chain data.
\item \textbf{Prysm} requires 324 MiB for 124,342 slots.
\item \textbf{Lighthouse} requires 403 MiB for 124,342 slots.
\item \textbf{Nimbus} requires 3.98 GiB for 124,295 slots.
\end{itemize}

The data might indicate that the Teku, Prysm, and Lighthouse client implement database pruning, i.e., by removing everything that is invalidated by finalization.\par

\subsection{Memory Metrics}
Figure \ref{img:mem} displays resident set size reported by the four clients. Again, the patterns are left uncommented. Notably, the Nimbus and Lighthouse clients appear to be most efficient with regard to memory usage, peaking at around 500 MiB in default operation mode. Prysm peaks at just below 1.3 GiB, Teku at a little less than 10 GiB.\par

\subsection{Networking Metrics}
\label{sec:perf:p2p}
Figure \ref{img:per} displays the peer count of every client during operation. There is not much to be commented on and just serves as a sanity check to rule out networking issues that could impact any of the other metrics.\par

Notably, there is a drastic drop in peers of the Nimbus client which, however, does not correlate to any of the synchronization metrics.\par

\section{Conclusion}
%Even though it's early to make any substantial conclusions and the clients are still undergoing very active development, two things cannot be ignored when looking at the data.

%First of all, it's difficult to compare Ethereum 1.x clients with Ethereum 2.0 clients due to the entirely different architecture and cryptography they require. However, from a practical perspective, these numbers \textit{do} matter for some -- if not many -- users out there. While being fully aware of the ongoing active development in the Ethereum 2.0 client landscape, we should be encouraged to double down research on increasing these numbers by \textit{orders of magnitude} rather than \textit{some percentage}.\par

%Being able to verify a block in 10 milliseconds or less (hypothetical number) is not only faster and more convenient for users but also improves the node's stability and the robustness of the network against denial-of-service attacks.\par

%Secondly, a huge shoutout to the Sigma Prime team positioning their Lighthouse client on top in every metric collected. It shall be noted that Ethereum 2.0 is not a race or competition but rather a massive collaborative effort to design and implement a new blockchain protocol. While this is not about calling out a winner, we should be encouraged to learn from the Sigma Prime team's design decisions and the Lighthouse architecture to potentially improve other Ethereum 2.0 node implementations.\par

\vspace{\fill}

\section*{Note}
The author is not affiliated with any of the teams implementing an Ethereum 2.0 client. The author is independently funded through the Ethereum Foundation's Ecosystem Support Program\footnote{\href{https://esp.ethereum.foundation}{esp.ethereum.foundation}} and the Goerli Testnet Initiative\footnote{\href{https://goerli.net}{goerli.net}}.\par

The author is not speaking on behalf of any organization.\par

\texttt{:)}
\end{document}
